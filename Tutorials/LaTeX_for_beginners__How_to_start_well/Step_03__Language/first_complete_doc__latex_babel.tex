
%%        %%
% PREAMBLE %
% ======== %



% Document class
% --------------
\documentclass[11pt, a4paper, english]{article}

% Font
% ----
\usepackage{lmodern}

\usepackage[utf8]{inputenc}
\usepackage[T1]{fontenc}

% Language and typography
% -----------------------
\usepackage[main=english, french]{babel}

\usepackage[autostyle=true]{csquotes}

\usepackage[babel=true]{microtype}

% Lorem ipsum
% -----------
\usepackage{lipsum}



%%        %%
% DOCUMENT %
% ======== %



% Metadata
% --------
\author{Arkh42}

\title{First complete document: \LaTeX{} compiler and \texttt{babel} package}



% Content
% -------
\begin{document}
	
	
	\maketitle
	

	All examples are based on the Latin Modern Roman font family.
	Text is generated thanks to Lorem Ipsum.
	
	
	\section{Font families examples}
	
		Normal font example --- \lipsum[1]
		
		Slanted font example (rarely used) --- \textsl{\lipsum[1]}
		
		Italic font example --- \textit{\lipsum[1]}
		
		Bold font example --- \textbf{\lipsum[1]}
		
		Italic bold font example --- \textbf{\textit{\lipsum[1]}}
		
		Sans serif font example --- \textsf{\lipsum[1]}
		
		Teletype font example (monospace) --- \texttt{\lipsum[1]}
		
		Small capital font example --- \textsc{\lipsum[1]}
		
	
	\section{Language selection and typography}
	
		
		\subsection{In English}
		
			A list in English, with some typographical rules which must be observed:
			\begin{itemize}
				\item item 1,
				\item item 2, and
				\item item $n$.
			\end{itemize}
		
			In the previous example, one can see that the first paragraph is not indented; in addition, there are no white spaces before typographical symbols such as colons and semi-colons.
			
			Here is a use of the \texttt{csquotes} package in conjunction with \enquote{babel}. The English quotes are used. 
	
	
		%\selectlanguage{french}%		Command
		\begin{otherlanguage}{french}%	Environment
		\subsection{Un peu de français}
		
			Une liste en français, comprenant les règles typographiques qui doivent être suivie:
			\begin{itemize}
				\item item 1;
				\item item 2;
				\item item $n$.
			\end{itemize}
			
			Contrairement à une idée reçue très répandue, il n'y a pas d'interligne entre les paragraphes en français; par ailleurs, en français, une espace fine se place devant les symboles typographiques tels que les points-virgules.
			
			Remarque importante: contrairement à l'anglais, tous les paragraphes sont indentés en français; malheureusement, le language principal de ce document étant l'anglais, le package \textit{babel} ne respecte pas l'ensemble des règles typographiques.
			
			Voici un exemple d'utilisation du package \texttt{csquotes} en conjonction avec \enquote{babel}. L'usage des guillemets français, et des espaces, est respecté.
		\end{otherlanguage}
		

\end{document}