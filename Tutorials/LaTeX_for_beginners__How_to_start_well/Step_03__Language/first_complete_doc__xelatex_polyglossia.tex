
%%        %%
% PREAMBLE %
% ======== %



% Document class
% --------------
\documentclass[11pt, a4paper, english]{article}

% Font
% ----
\usepackage{fontspec}
\setmainfont[%
	SmallCapsFont={* Caps},%	enable small capital font family
	SlantedFont={* Slanted},%	enable slanted font family
]{Latin Modern Roman}

% Language and typography
% -----------------------
\usepackage{polyglossia}
\setdefaultlanguage[variant=british]{english}
\setotherlanguage{french}

\usepackage[autostyle=true]{csquotes}

\usepackage{microtype}

% Lorem ipsum
% -----------
\usepackage{lipsum}



%%        %%
% DOCUMENT %
% ======== %



% Metadata
% --------
\author{Arkh42}

\title{First complete document: Xe\LaTeX{} compiler and \texttt{polyglossia} package}



% Content
% -------
\begin{document}
	
	
	\maketitle
	
	
	All examples are based on the Latin Modern Roman font family.
	Text is generated thanks to Lorem Ipsum.
	
	
	\section{Font families examples}
	
		Normal font example --- \lipsum[1]
		
		Slanted font example (rarely used) --- \textsl{\lipsum[1]}
		
		Italic font example --- \textit{\lipsum[1]}
		
		Bold font example --- \textbf{\lipsum[1]}
		
		Italic bold font example --- \textbf{\textit{\lipsum[1]}}
		
		Sans serif font example --- \textsf{\lipsum[1]}
		
		Teletype font example (monospace) --- \texttt{\lipsum[1]}
		
		Small capital font example --- \textsc{\lipsum[1]}
	
	
	\section{Language selection and typography}
	
	
		\subsection{In English}
		
		A list in English, with some typographical rules which must be observed:
		\begin{itemize}
			\item item 1,
			\item item 2, and
			\item item $n$.
		\end{itemize}
		
		In the previous example, one can see that the first paragraph is not indented; in addition, there are no white spaces before typographical symbols such as colons and semi-colons.
		
		Here is a use of the \texttt{csquotes} package in conjunction with \enquote{polyglossia}. The English quotes are used. 
	
	
	%\selectlanguage{french}%		Command
	\begin{french}%	Environment
		
		\subsection{Un peu de français}
		
			Une liste en français, comprenant les règles typographiques qui doivent être suivie:
			\begin{itemize}
				\item item 1;
				\item item 2;
				\item item $n$.
			\end{itemize}
			
			Contrairement à une idée reçue très répandue, il n'y a pas d'interligne entre les paragraphes en français; par ailleurs, en français, une espace fine se place devant les symboles typographiques tels que les points-virgules.
			
			Remarque importante: \textit{polyglossia} semble légèrement plus performant que \textit{babel} puisque l'indentation du premier paragraphe est respectée. Par contre, aucun des deux ne respecte l'utilisation de tirets.
			
			Voici un exemple d'utilisation du package \texttt{csquotes} en conjonction avec \enquote{polyglossia}. L'usage des guillemets français, et des espaces, est respecté.
			
			Remarque importante: il semble y avoir un problème lorsque l'on compile avec \textit{xelatex} (cf. espaces étranges autour du mot polyglossia dans le paragraphe précédent).
	\end{french}
	
	
\end{document}
